\documentclass[11pt]{beamer} %handout
\usepackage[utf8]{inputenc} 
\usetheme{Madrid}
\usepackage{amsmath}
\usepackage{tikz-cd}
\usepackage{adjustbox}
\usepackage{amsmath,amsthm,amssymb,mathtools,amsfonts}
\usepackage[italian]{babel}
\usepackage{pgfornament}
\usepackage{pgfplots}
\usepackage{mathrsfs}


\pgfplotsset{compat=1.15}
\usetikzlibrary{arrows}

\author[Chiara Molinari]{Chiara Molinari\\[10mm] Relatore: Prof. Luigi Ambrosio}
\title[Il Teorema di Davies]{Costruzioni paradossali in Teoria della Misura: il Teorema di Davies}
\date{10 giugno 2022}

\newcommand{\ra}{\rightarrow}
\newcommand{\Ra}{\Rightarrow}
\newcommand{\C}{\mathbb{C}}
\newcommand{\R}{\mathbb{R}}
\newcommand{\N}{\mathbb{N}}
\newcommand{\Q}{\mathbb{Q}}
\newcommand{\Z}{\mathbb{Z}}
\renewcommand{\S}{\mathbb{S}}
\renewcommand{\P}{\mathbb{P}}
\newcommand{\parti}{\mathcal{P}}
\newcommand{\E}{\mathcal{E}}
\newcommand{\A}{\mathcal{A}}
\newcommand{\fa}{\forall}
\newcommand{\fo}{\forall}
\newcommand{\e}{\varepsilon}
\newcommand{\<}{\langle}
\renewcommand{\>}{\rangle}

% Bold math in titles
\makeatletter%
\DeclareRobustCommand*{\bfseries}{%
	\not@math@alphabet\bfseries\mathbf
	\fontseries\bfdefault\selectfont
	\boldmath
}
\makeatother

\theoremstyle{theorem}
\newtheorem{teo}{Teorema}

\theoremstyle{theorem}
\newtheorem{teorema}{Teorema}
\newtheorem{oss}[teo]{Osservazione}
\theoremstyle{theorem}
\newtheorem{proposizione}{Proposizione}

\theoremstyle{theorem}
\newtheorem{corollario}{Corollario}

\newtheorem{defin}[teo]{Definizione}

\theoremstyle{theorem}
\newtheorem{lem}{Lemma}

\usefonttheme{professionalfonts} %questo coso diabilita la gestione di font di beamer



\begin{document}%%%%%%%%%%%%%%%%%%%%%%%%
\begin{frame}
	\maketitle
\end{frame}	

\begin{frame}[fragile]
	\frametitle{Introduzione: paradossi in teoria della misura}
	Talvolta il comportamento della misura di Lebesgue può essere molto controintuitivo: un esempio famoso è il paradosso di Banach-Tarski.\\
	\pause
	Si possono ottenere risultati paradossali anche restringendosi a considerare insiemi misurabili, ottenuti con metodi costruttivi che non richiedono l’uso dell'assioma della scelta.\\
	\pause
	
	\begin{teo}[Davies, 1951] Dato un insieme misurabile $A \subseteq \R^2$, esiste un insieme di rette $L$ tale che:\\
		\begin{itemize}	
			\item per ogni punto di $A$ passa una retta di $L$;\\
			\item $\mu(A)=\mu(L^*)$, dove $\mu$ è misura di Lebesgue e $L^*$ il sottoinsieme di $\R^2$ coperto dalle rette di $L$.\\
		\end{itemize}
	\end{teo}
 
\end{frame}


%\begin{frame}
%	La dimostrazione originale utilizzava il seguente lemma.\\

%	Sia $R$ un parallelogramma $R$ e $K$ un sottoinsieme chiuso (misurabile). Allora dato $\varepsilon>0$, esiste una collezioni finita (numerabile) di parallelogrammi $\{P_i\}_{i \in I}$ contenuti in $R$, con lati paralleli a quelli di $R$ tali che:\\
%	\begin{enumerate}[i]
	%		\item $K \subseteq \bigcup_i P_i$;	
	%		\item $|\bigcup_i P_i \setminus K|< \varepsilon$.
	%	\end{enumerate}
%\end{frame}


\begin{frame}[fragile]
\frametitle{Introduzione}	
		Nel 2001, viene dimostrata una generalizzazione del teorema, nel caso in cui $\mu$ è una misura boreliana $\sigma$-finita.\\
		Nella tesi è dimostrata tale generalizzazione, attraverso i seguenti passi intermedi: \pause
		\begin{itemize}
			\item la dimostrazione nel caso della misura di Lebesgue, introducendo la formulazione duale del problema e alcune costruzioni geometriche nel piano;\\ \pause
			\item lo sviluppo della teoria di Suslin;\\ \pause
			\item la dimostrazione nel caso di $\mu$ generica boreliana $\sigma$-finita.
		\end{itemize}
\end{frame}

\begin{frame}[fragile]
	\frametitle{Formulazione duale}
	Esiste una corrispondenza tra le rette in $\P^2 \R$ e i punti in $\P^2\R$: alla retta di equazione $ax_0+bx_1+cx_2=0$ associo il punto $[a,b,c]$.\\
	\pause
	Si consideri il piano proiettivo reale $\P^2 \R$, realizzato come quoziente di $\S^2$. Una misura su $\S^2$ induce tramite la proiezione al quoziente una misura su $\P^2\R$.\\
	\pause
	Possiamo così definire una misura sulle rette proiettive. Se una retta è affine, considero la sua equazione omogeneizzata.\\
	L'usuale misura di area su $\S^2$, induce una naturale misura $\tilde \theta$ sulle rette.\\
	\pause
	In modo simile, diciamo che un insieme di rette è aperto/compatto/boreliano/misurabile se l'insieme di punti corrispondente nel duale lo è.	
\end{frame}

\begin{frame}
	\frametitle{Formulazione duale}
	
	\begin{block}{Osservazione}
		Un insieme di rette passanti per un punto $P \in \P^2 \R$, è rappresentato nel duale da punti appartenenti a una retta $\ell_P$.\\
	\end{block}
	\pause
	\medskip
	Fissato un punto $P$ nel piano, la misura $\tilde \theta$ delle rette passanti per esso è sempre nulla.\\
	\pause
	Usando l'osservazione, posso comunque definire una misura sulle rette passanti per $P$: è quella indotta dalla naturale misura su $\ell_P$ (omeomorfo a $\S^1$ quozientato).\\
\end{frame}	

	

\begin{frame}[fragile]
	\frametitle{Definizioni}

	\begin{defin}
	Un insieme è di \textbf{prima categoria} se unione numerabile di insiemi mai densi (cioè con chiusura a parte interna vuota).\\ \pause
	Un insieme è \textbf{residuale} se il suo complementare è di prima categoria. 
	\end{defin}
	\pause
	\begin{defin}
	Dato un insieme di rette nel piano $L$, indichiamo con $L^*$ l'insieme di punti coperti da rette di $L$.\\
	\pause
	Similmente, dato un insieme di punti nel piano $A$, indichiamo con $A^*$ l'insieme delle rette passanti per punti di $A$.
	\end{defin}

\end{frame}

\begin{frame}[fragile]

\begin{lemma}[1]
	Sia $A$ aperto del piano e sia $x$ un punto che non appartiene ad $A$. Allora esiste un insieme boreliano di rette $L$ tale che:\\
	\begin{itemize}
		\item per ogni punto $p \in A$, l'insieme delle rette di $L$ passanti per $p$ è residuale;
		\item $L^{*} \backslash A$ interseca ogni retta per $x$ in un insieme di misura di Lebesgue nulla.\\
	\end{itemize}
\end{lemma}
	\pause
	Questo è un rafforzamento del teorema di Davies per la misura di Lebesgue.\\
	\pause
	\begin{itemize}
		\item Si mostra che condizione $A$ aperto non è restrittiva.\\
		\pause
		\item Viene aggiunta una condizione topologica su $L$. Questa implica che $L$ copre $A$ per il teorema di Baire.\\
		\pause
		\item Per Fubini, il sottoinsieme del piano $L^* \setminus A$ ha misura di Lebesgue nulla se e solo se, dato un punto $x$, \emph{quasi} ogni retta per $x$ interseca $L^* \setminus A$ in un insieme di misura lineare nulla. Togliendo il \emph{quasi}, la tesi è più forte.\\
	\end{itemize}

\end{frame}



\begin{frame}
	Enunciamo così la versione duale del precedente lemma.
	
	\begin{lemma}[2]
	Sia $L$ un insieme aperto di rette e sia $X$ un retta non appartenente a $L$. Allora esiste un insieme boreliano di punti $A$ per cui:\\
	\begin{itemize}
		\item ogni retta di $L$ interseca $A$ in un insieme residuale;\\
		\item per ogni punto di $X$ passano trascurabili rette di $A^{*} \backslash L$.\\
	\end{itemize}

	\medskip
	
	\end{lemma}
	\pause
	Il primo passo della dimostrazione è assumere che $X$ sia la retta all'infinito.\\
	\pause
	In seguito la dimostrazione è piuttosto tecnica e vengono usate costruzioni geometriche che induttivamente permettono di costruire collezioni di parallelogrammi nel piano, con certe caratteristiche.\\
\end{frame}	



\begin{frame}[fragile]
	\frametitle{Costruzioni per parallelogrammi}
	Una di queste è la costruzione geometrica detta ``venetian blind", che dato un parallelogramma $P$, gli associa una collezione finita di parallelogrammi contenuti in $P$, nel seguente modo.
	\pause
	\begin{center}
		\includegraphics[width=0.9\columnwidth]{prova1.png}
	\end{center}
	\pause
	Ogni retta in direzione appartenente a $D_1$ che interseca $P$ interseca anche uno dei nuovi parallelogrammi.\\
	\pause
	La misura della proiezione dell'unione dei parallelogrammi sulla retta $A_{1} A_{4}$ è al più $2 \cdot \left|A_{1} A_{4}\right|$ in ogni direzione in $D_2$.
\end{frame}


\begin{frame}[fragile]
	\frametitle{Costruzioni per parallelogrammi}
	
	\begin{center}
	\includegraphics[width=0.9\columnwidth]{prova2.png}
	\end{center}

	Induttivamente, all'$n$-esimo passo, una retta in direzione in $D_1$ che interseca $P$, interseca uno dei parallelogrammi dell'$n$-esimo passo.\\
	Inoltre la misura della proiezione dell'unione dei parallelogrammi 
	sulla retta $A_{1} A_{4}$ è al più $2 \cdot\left|A_{1} A_{4}\right|$ in ogni direzione in $D_2,...,D_{n+1}$.\\

\end{frame}



\begin{frame}
	\frametitle{Costruzioni per parallelogrammi}
	\begin{lemma}[2]
		Sia $L$ un insieme aperto di rette. Allora esiste un insieme boreliano di punti $A$ per cui:\\
		\begin{itemize}
			\item ogni retta di $L$ interseca $A$ in un insieme residuale;\\
			\item in ogni direzione ci sono trascurabili rette che intersecano $A$ e non appartengono a $L$.\\
		\end{itemize}
	\end{lemma}
	
	\pause
Si costruisce $A$ esplicitamente, tramite intersezioni e unioni applicate a una collezione numerabile di parallelogrammi ottenuti con la procedura di prima.\\ \pause
Per quasi ogni direzione $d$, si costruisce un insieme $A_{d} \subseteq A$ con questa proprietà: ogni una retta non in $L$ e in direzione $d$ che interseca $A$ interseca anche $A_{d}$.\\ \pause
Si conclude mostrando che la misura della proiezione di $A_{d}$ in direzione $d$ è arbitrariamente piccola.\\ \pause
La condizione di residualità è piuttosto diretta.
\end{frame}	


\begin{frame}
	\frametitle{Generalizzazione del teorema}
	
	\begin{teo}
		Sia $A \subset \mathbb{R}^{2}$ misurabile e $\mu$ una misura boreliana $\sigma$-finita nel piano. Allora esiste un insieme boreliano di rette $L$ tale che: \\
		- $L$ contiene un insieme residuale di rette per ogni punto di $A$;\\
		- $\mu(A)=\mu(L^*)$. \\ \pause
		Similmente, data una misura boreliana $\tilde \mu$ $\sigma$-finita sull'insieme delle rette nel piano, per un insieme misurabile di rette $L$, esiste un insieme di punti $A$ tale che ogni retta di $L$ interseca $A$ in un insieme residuale e $\tilde \mu(A^*)=\tilde \mu(L)$.
	\end{teo}
	\pause
	Ci soffermiamo solo sul primo enunciato. La prima verifica è che dato $L$ insieme di rette boreliano, allora $L^*$ è misurabile rispetto a qualsiasi misura $\mu$ boreliana finita.
\end{frame}

\begin{frame}
	\frametitle{Misurabilità di $L^*$}
	Introduciamo alcune definizioni generali.

\begin{defin}
	Sia $X$ un insieme non vuoto e sia $\mathcal{E}$ una collezione di suoi sottoinsiemi. \pause Diciamo che un insieme $A$ è analitico o di Suslin se è nella forma
	$$
	A=\bigcup_{\left(n_{i}\right) \in \mathbb{N}^{\infty}} \bigcap_{k=1}^{\infty} A_{n_{1}, \ldots, n_{k}} .
	$$
	per opportuni $A_{n_{1}, \ldots, n_{k}} \in \E$.\\ \pause
	Questa è detta operazione di Suslin.\\
	La collezione degli insiemi di questo tipo e l'insieme vuoto è indicata con $S(\mathcal{E})$.
\end{defin}	

\end{frame}

\begin{frame}[fragile]
\frametitle{Misurabilità di $L^*$}	
Valgono i seguenti teoremi generali. \pause

\begin{teo}
	\begin{enumerate}[i]
		\item $S(S(\mathcal{E}))=S(\mathcal{E})$\\ \pause
		\item Se il complementare di ogni insieme in $\E$ appartiene a $S(\E)$ e $\varnothing \in \E$, la $\sigma$-algebra generata da $\mathcal{E}$ è contenuta in $S(\mathcal{E}).$
	\end{enumerate}
\end{teo}
\pause
Siccome $L$ insieme boreliano, applicando il teorema con $\E$ la classe dei compatti di $\P^2 \R$, si ha
$$L=\bigcup_{{(n_i)}\in{\N}^\infty}\bigcap_{k=1}^\infty K_{n_1,\ldots,n_k}$$
con $K_{n_1,\ldots,n_k}\in {\cal E}$. Posso anche supporre che l'intersezione sia decrescente.

\end{frame}

\begin{frame}
\frametitle{Misurabilità di $L^*$}	
Si mostra che vale anche
$$L^*=\bigcup_{{ (n_i)}\in{\N}^\infty}\bigcap_{k=1}^\infty K_{n_1,\ldots,n_k}^*.$$ 
\pause
Si mostra che dato $K$ compatto di rette, allora $K^*$ analitico. \pause
Ricordando che $S(S(\E))=S(\E)$, abbiamo quindi che $L^*$ analitico. \pause Concludiamo applicando il seguente teorema. \\
\pause
\begin{teo}
 	Detta $\mu$ una misura finita. Se $\E$ famiglia di insiemi misurabili chiusa per unioni finite e intersezioni numerabili, ogni insieme in $S(\E)$ è $\mu$-misurabile.
\end{teo}	
	
	
\end{frame}


\begin{frame}[fragile]

\frametitle{Generalizzazione del teorema}
Dato $A$ misurabile, cerchiamo quindi un insieme boreliano $L$ che copra $A$ con le caratteristiche richieste. \\ \pause
\medskip
\structure{\textbf{Dimostrazione}}
	\begin{itemize}
		\item Passo 1: possiamo assumere che $\mu$ abbia supporto compatto $K$ disgiunto da $A$.\\ \pause
		\item Passo 2: possiamo assumere che $A=B(x, r) \backslash\{x\}$ e $B(x, 2 r)$ sia disgiunta da $K$. Per brevità, assumiamo $A=B(0,1) \backslash\{0\}$ e $B(0,2) \cap K=\varnothing$. \pause
		\item Passo 3: possiamo assumere che $\mu$ sia finita. \pause
		\item Passo 4: per concludere, applichiamo il lemma 1 (teorema di Davies per la misura di Lebesgue).
	\end{itemize}
\end{frame}


\begin{frame}
\frametitle{Passo 4}
Applicando il lemma 1 ad $A$ e $x=0$, otteniamo l'insieme boreliano di rette $M$. \pause \\
\begin{itemize}
	\item $M$ contiene un insieme residuale di rette per ogni punto di $A$;\\
	\item $M^*\setminus A$ interseca ogni retta per $0$ in un insieme di misura di Lebesgue nulla.
\end{itemize} \pause
	L'insieme $L$ di rette cercato è $L=tM$, per un certo $1 \leq t \leq 2$ (che si dimostra esistere). \\ \pause
	$L$ così definito è ancora boreliano, da cui $L^*$ misurabile. Si mostra che soddisfa la condizione di residualità e che $\mu (L^*)= 0= \mu(A)$.
\end{frame}

\begin{frame}
	\frametitle{Bibliografia}
	
	\structure{\textbf{Bibliografia}}
	\begin{itemize}
		\item \emph{On accessibility of plane sets and differentiation
			of functions of two real variables}, R. O. Davies
		\item \emph{How to make Davies' Theorem visible}, M. Cs{\"o}rnyei
		\item \emph{On the visibility of invisible sets}, M. Cs{\"o}rnyei
		\item \emph{Measure Theory}, V.I. Bogachev\\
		\end{itemize}	
	
	\medskip
	\begin{center}
	\structure{\textbf{Grazie per l'attenzione}}
	\end{center}
\end{frame}



\end{document}%%%%%%%%%%%%%%%%%%%%%%%%
