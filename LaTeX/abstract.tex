

\documentclass[11pt]{article}

\usepackage[utf8]{inputenc} 
\usepackage{graphicx}
\usepackage{amsmath,amsthm,amssymb,mathtools,amsfonts}
\usepackage{enumitem}
\usepackage[italian]{babel}


\title{Costruzioni paradossali in Teoria della Misura:\\
	il Teorema di Davies}
\author{Chiara Molinari}
\date{10 giugno 2022} % Activate to display a given date or no date (if empty),
         % otherwise the current date is printed 

\setlength{\parindent}{0pt}
\pagenumbering{arabic}

\newcommand{\ra}{\rightarrow}
\newcommand{\Ra}{\Rightarrow}
\newcommand{\LRa}{\Leftrightarrow}
\newcommand{\C}{\mathbb{C}}
\newcommand{\R}{\mathbb{R}}
\newcommand{\N}{\mathbb{N}}
\newcommand{\Q}{\mathbb{Q}}
\newcommand{\Z}{\mathbb{Z}}
\renewcommand{\P}{\mathbb{P}}
\renewcommand{\S}{\mathbb{S}}
\newcommand{\D}{\mathbb{D}}
\newcommand{\parti}{\mathcal{P}}
\newcommand{\fa}{\forall}
\newcommand{\fo}{\forall}
\newcommand{\e}{\varepsilon}
\newcommand{\<}{\langle}
\renewcommand{\>}{\rangle}
\newcommand{\E}{\mathcal{E}}
\newcommand{\A}{\mathcal{A}}

\DeclarePairedDelimiter{\abs}{\lvert}{\rvert}
\DeclarePairedDelimiter{\norm}{\lVert}{\rVert}

\newtheorem{teo}{Teorema}[]
\newtheorem{lemma}[teo]{Lemma}
\newtheorem{defin}[teo]{Definizione}
\newtheorem{cor}[teo]{Corollario}
\newtheorem{oss}[teo]{Osservazione}
\newtheorem{es}[teo]{Esempio}
\newtheorem{prop}[teo]{Proposizione}
\newtheorem{eserc}[teo]{Esercizio}


\begin{document}
	
\maketitle

\section*{Abstract}
	
	Nel 1952 R. O. Davies dimostrò il seguente enunciato, apparentemente paradossale. Dato un insieme misurabile $A$ nel piano, esiste un insieme di rette $L$ tali che:\\
	- per ogni punto di $A$ passa una retta di $L$;\\
	- l'insieme di punti coperto dalle rette di $L$ ha la stessa misura di Lebesgue di $A$.\\
	L'obiettivo è dimostrare una generalizzazione del teorema per una generica misura boreliana $\sigma$-finita.\\
	Il primo passo è mostrare una versione del teorema di Davies nel caso della misura della Lebesgue e $A$ aperto. Lo sviluppo di questa sezione richiede l'introduzione di una nozione di dualità (cioè una corrispondenza tra rette e punti) e una formulazione duale del problema. Inoltre sono usate raffinate costruzioni geometriche nel piano, che permettono di costruire collezioni infinite di parallelogrammi con determinate caratteristiche.\\
	Tale versione del teorema di Davies, grazie anche all'introduzione di alcuni concetti della teoria degli insiemi di Suslin, consente la generalizzazione cercata.
	
	
	

\end{document}
