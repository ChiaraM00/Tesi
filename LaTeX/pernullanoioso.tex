

\documentclass[11pt]{article}

\usepackage[utf8]{inputenc} 
\usepackage{graphicx}
\usepackage{amsmath,amsthm,amssymb,mathtools,amsfonts}
\usepackage{enumitem}
\usepackage[italian]{babel}
\usepackage{geometry}
\geometry{
	a4paper,
	total={160mm,257mm},
	left=25mm,
	top=30mm,
}

\setlength{\parindent}{0pt}
\pagenumbering{arabic}

\newcommand{\ra}{\rightarrow}
\newcommand{\Ra}{\Rightarrow}
\newcommand{\LRa}{\Leftrightarrow}
\newcommand{\C}{\mathbb{C}}
\newcommand{\R}{\mathbb{R}}
\newcommand{\N}{\mathbb{N}}
\newcommand{\Q}{\mathbb{Q}}
\newcommand{\Z}{\mathbb{Z}}
\renewcommand{\P}{\mathbb{P}}
\renewcommand{\S}{\mathbb{S}}
\newcommand{\D}{\mathbb{D}}
\newcommand{\parti}{\mathcal{P}}
\newcommand{\fa}{\forall}
\newcommand{\fo}{\forall}
\newcommand{\e}{\varepsilon}
\newcommand{\<}{\langle}
\renewcommand{\>}{\rangle}
\newcommand{\E}{\mathcal{E}}
\newcommand{\A}{\mathcal{A}}


\begin{document}
	

	\section*{Costruzioni paradossali in Teoria della Misura: il Teorema di Davies}
	Questo è un riassunto della mia tesi, in cui spiego cosa ho fatto. I concetti talvolta sono molto elementari e intutivi, ma il risultato finale non lo è.\\
	
	\textbf{Misura nel piano}\\
	Intuitivamente, la misura di un insieme è la sua area.\\
	La misura di area è detta misura di Lebesgue nel piano. La misura di Lebesgue di un rettangolo è uguale a "base per altezza". Inoltre la misura di Lebesgue di una retta o di un punto nel piano sono nulle.\\
	Posso generalizzare questa nozione intuitiva, definendo delle misure distribuite diversamente. Ad esempio, per un rettangolo, la sua misura potrebbe non essere più "base per altezza", ma qualcosa di molto strano.\\
	Comunque c'è un po' di coerenza: se ho un rettangolo grande che contiene un rettangolo piccolo, la misura del rettangolo grande deve essere maggiore o uguale di quella del rettangolo piccolo.\\

	Segue la definizione formale di misura (con piccole semplificazioni), dove sono spiegate le proprietà di coerenza. Se non si capisce, pazienza, si può andare alla scritta in grassetto.\\
	Dato un insieme $X$ (può essere ad esempio il piano, ma anche qualcosa di molto più complicato), una misura è una funzione $\mu$ che associa a ogni sottoinsieme di $X$ un numero reale e che soddisfa le seguenti proprietà:\\
	- non-negatività: per ogni $E$ sottoinsieme di $X$ vale $\mu(E)\geq0$.\\
	- insieme vuoto: $ \mu (\varnothing )=0$.\\
	- additività: dati due insiemi $A,B$ disgiunti, $\mu(A \cup B)= \mu (A)+ \mu(B)$. (Vale anche per unioni di infiniti insiemi, non solo due).\\
	
	Queste proprietà sono molto sensate perché sono intuitive se pensiamo alla misura di Lebsegue in una/due/tre dimensioni (cioè per le misure di lunghezza, area, volume).\\
	Un altro esempio è quello delle misure di probabilità, cioè misure in cui si chiede che la misura di tutto $X$ sia 1.\\
	Ad esempio $X$ può essere l'insieme di esiti di un lancio di dado. Ogni esito ha misura 1/6. Pensandoci un po', si vede che le proprietà scritte sopra sono tutte verificate.\\

	\textbf{Arriviamo alla tesi vera e propria.}\\
	Nel 1952 R. O. Davies dimostrò il seguente enunciato, apparentemente paradossale. Dato un insieme $A$ nel piano, esiste un insieme di rette $L$ tali che:\\
	- per ogni punto di $A$ passa una retta di $L$;\\
	- l'insieme di punti coperto dalle rette di $L$ ha la stessa misura di Lebesgue (cioé la stessa area) di $A$.\\
	L'obiettivo è dimostrare una generalizzazione del teorema per una generica misura nel piano.\\
	Il primo passo è mostrare una versione del teorema di Davies nel caso della misura della Lebesgue e $A$ con certe proprietà. Lo sviluppo di questa sezione richiede l'introduzione di una nozione di dualità (cioè una corrispondenza tra rette e punti). Inoltre sono usate costruzioni geometriche nel piano, che permettono di costruire collezioni infinite di parallelogrammi con determinate caratteristiche.\\
	Il secondo passo è generalizzare come dicevo sopra.
	
	
	
	
\end{document}
