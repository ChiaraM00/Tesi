\documentclass[11pt]{beamer} %handout
\usepackage[utf8]{inputenc} 
\usetheme{Madrid}
\usepackage{amsmath}
\usepackage{physics}
\usepackage{tikz-cd}
\usepackage{tikz-cd}
\usepackage{adjustbox}
\usepackage{amsmath,amsthm,amssymb,mathtools,amsfonts}
\usepackage[italian]{babel}
\usepackage{pgfornament}

\author{Chiara Molinari}
\title{Costruzioni paradossali in Teoria della Misura: il Teorema di Davies}
\date{29 aprile 2022}

\newcommand{\ra}{\rightarrow}
\newcommand{\Ra}{\Rightarrow}
\newcommand{\C}{\mathbb{C}}
\newcommand{\R}{\mathbb{R}}
\newcommand{\N}{\mathbb{N}}
\newcommand{\Q}{\mathbb{Q}}
\newcommand{\Z}{\mathbb{Z}}
\renewcommand{\S}{\mathbb{S}}
\renewcommand{\P}{\mathbb{P}}
\newcommand{\parti}{\mathcal{P}}
\newcommand{\E}{\mathcal{E}}
\newcommand{\A}{\mathcal{A}}
\newcommand{\fa}{\forall}
\newcommand{\fo}{\forall}
\newcommand{\e}{\varepsilon}
\newcommand{\<}{\langle}
\renewcommand{\>}{\rangle}

% Bold math in titles
\makeatletter%
\DeclareRobustCommand*{\bfseries}{%
	\not@math@alphabet\bfseries\mathbf
	\fontseries\bfdefault\selectfont
	\boldmath
}
\makeatother

\theoremstyle{theorem}
\newtheorem{teo}{Teorema}

\theoremstyle{theorem}
\newtheorem{teorema}{Teorema}

\theoremstyle{theorem}
\newtheorem{proposizione}{Proposizione}

\theoremstyle{theorem}
\newtheorem{corollario}{Corollario}

\newtheorem{defin}[teo]{Definizione}

\theoremstyle{theorem}
\newtheorem{lem}{Lemma}

\usefonttheme{professionalfonts} %questo coso diabilita la gestione di font di beamer




\begin{document}%%%%%%%%%%%%%%%%%%%%%%%%

\begin{frame}
	\maketitle
\end{frame}


\begin{frame}[fragile]
	\structure{\textbf{Teorema (Davies, 1951):}} dato un insieme misurabile $A \subseteq \R^2$, con $\mu(E)< \infty$ (misura di Lebesgue), esiste un insieme di linee $L$ tale che:\\
	(i) per ogni punto di $E$ passi una linea di $L$;\\
	(ii) $\mu(L^*)=\mu(E)$;\\
	dove con $L^*$ si indica il sottoinsieme di $\R^2$ coperto dalle rette di $L$.
	
	\medskip
	
	Nel 2001, M. Csornyei dimostra un rafforzamento del teorema e una sua generalizzazione al caso di una generica misura boreliana $\sigma$-finita. 
	
	\medskip
	 
	In questo colloquio:\\
	\begin{itemize}
		\item accenniamo alla dimostrazione originale;\\
		\item mostriamo alcune idee utilizzate da Csornyei per dimostrare il teorema rafforzato (la costruizione geometrica "venetian blind" e la formulazione duale del problema);\\
		\item tratteremo il problema della misurabilità di $L^*$ come sopra;\\
		\item dimostreremo il caso per $\mu$ generica boreliana $\sigma$-finita.
	\end{itemize}

\end{frame}
	
\begin{frame}
	
	
	La dimostrazione originale utilizzava il seguente lemma.\\
	
	Sia $R$ un parallelogramma $R$ e $K$ un sottoinsieme chiuso (misurabile). Allora dato $\varepsilon>0$, esiste una collezioni finita (numerabile) di parallelogrammi $\{P_i\}_{i \in I}$ contenuti in $R$, con lati paralleli a quelli di $R$ tali che:\\
	\begin{enumerate}[i]
		\item $K \subseteq \bigcup_i P_i$;	
		\item $|\bigcup_i P_i \setminus K|< \varepsilon$.
	\end{enumerate}
	

\end{frame}


\begin{frame}
	\frametitle{Formulazione duale}
	Esiste una corrispondenza tra le rette in $\P^2 \R$ e i punti in $\P^2\R$: alla retta di equazione $ax_0+bx_1+cx_2=0$ associo il punto $[a,b,c]$.\\
	\pause
	Si consideri il piano proiettivo reale $\P^2 \R$, realizzato come quoziente di $\S^2$. Una misura su $S^2$ induce tramite la proiezione al quoziente una misura su $\P^2\R$.\\
	\pause
	Possiamo così definire una misura sulle rette proiettive. Se una retta è affine, considero la sua equazione omogeneizzata.\\
	L'usuale misura di area su $\S^2$, induce una naturale misura $\theta$ sulle rette.\\
	\pause
	In modo simile, diciamo che un insieme di rette è aperto/compatto/misurabile... se l'insieme di punti corrispondente nel duale lo è.

	
\end{frame}

\begin{frame}
	\frametitle{Formulazione duale}
	
	\begin{block}{Osservazione}
		Un'insieme di rette passanti per un punto $P$, è rappresentato nel duale da punti appartenenti a una retta $\ell_P$.\\
		\pause
		Infatti se $P=[a,b,c]$, una retta passante per $P$ soddisfa $ax_0+bx_1+cx_2=0$. Tale retta è rappresentata da $[x_0,x_1,x_2]$, che quindi appartiene alla retta $az_0+bz_1+cz_2=0$.\\
	\end{block}
	
	Fissato un punto $P$ nel piano, la misura $\theta$ delle rette passanti per esso è sempre nulla.\\
	\pause
	Usando l'osservazione, posso comunque definire una misura sulle rette passanti per $P$: è quella indotta dalla naturale misura su $\ell_P$, che è omeomorfo a $S^1$ quozientato.\\
\end{frame}	

	


\begin{frame}[fragile]
	Per enunciare il rafforzamento del teorema di Davies, diamo prima le seguenti definizioni.\\
	\pause	
	\begin{defin}
	Un insieme è della prima categoria se unione numerabile di insiemi mai densi (cioè con chiusura a parte interna vuota).\\
	Un insieme è residuo se il suo complementare è della prima categoria. 
	\end{defin}
	\pause
	\begin{defin}
	Dato un insieme di rette nel piano $L$, indichiamo con $L^*$ l'insieme di punti coperti da rette di $L$.\\
	\pause
	Similmente, dato un insieme di punti nel piano $A$, indichiamo con $A^*$ l'insieme delle rette passanti per punti di $A$.
	\end{defin}

\end{frame}

\begin{frame}[fragile]

\begin{lemma}
	Sia $A$ aperto del piano e sia $x$ un punto che non appartiene ad $A$. Allora esiste un insieme boreliano di rette $L$ tale che:\\
	\begin{itemize}
		\item L contiene un insieme residuo di rette per ogni punto di $A$;\\
		\item $L^{*} \backslash A$ interseca ogni retta per $x$ in un insieme di misura di Lebesgue nulla.\\
	\end{itemize}
\end{lemma}
	\pause
	Vediamo che è un rafforzamento.\\
	\pause
	\begin{itemize}
		\item Si mostra che condizione $A$ aperto non è restrittiva.\\
		\pause
		\item Aggiungiamo una condizione topologica su $L$ che implica che $L$ copre $A$.\\
		\pause
		\item Ricordiamo un insieme è di misura nulla se e solo se interseca $quasi$ ogni retta in un insieme di misura lineare nulla. Togliendo il $quasi$, la tesi è più forte.\\
	\end{itemize}

\end{frame}



\begin{frame}
	Enunciamo così il la versione duale del precedente lemma.
	
	\begin{lemma}
	Sia $L$ un insieme aperto di rette e sia $X$ un retta non appartenente a $L$. Allora esiste un insieme boreliano di punti $A$ per cui:\\
	\begin{itemize}
		\item ogni retta di $L$ interseca $A$ in un insieme residuo;\\
		\item per ogni punto di $X$ passano trascurabili rette di $A^{*} \backslash L$.\\
	\end{itemize}

	\end{lemma}
	\pause
	Il primo passo della dimostrazione è assumere che $X$ sia la retta all'infinito. La seconda tesi diventa quindi che in ogni direzione ci siano trascurabili rette che intersecano $A$ e che non appartengono a $L$.\\
	\pause
	D'ora in poi risulta difficile dare un'idea intuitiva della dimostrazione di questo lemma senza passare per teoremi tecnici.\\
	\pause
	Vengono ampiamente usate costruzioni geometriche ricorsive di parallelogrammi nel piano, con certe caratteristiche.\\
\end{frame}	



\begin{frame}[fragile]
	\frametitle{Costruzioni per parallelogrammi}
		A titolo di esempio, vediamo la costruzione geometrica detta "venetian blind", che dato un parallelogramma $P$, gli associa una collezione finita di parallelogrammi contenuti in $P$, nel seguente modo.
	\pause
	\begin{center}
		\includegraphics[width=0.5\columnwidth]{passo1.png}
	\end{center}
	
	Ogni retta in direzione appartenente a $D_1$ che interseca $P$ interseca anche uno dei nuovi parallelogrammi.\\
	\pause
	La misura delle proiezioni di ciascuno di questi parallelogrammi sulla retta $A_{1} A_{4}$ è al più $\left|A_{1} A_{4}\right|$ in ogni direzione in $D_2$. \pause Quindi la misura della proiezione dell'unione dei parallelogrammi è al più $2 \cdot\left|A_{1} A_{4}\right|$.\\
\end{frame}


\begin{frame}
	\frametitle{Costruzioni per parallelogrammi}
	\begin{lemma}
		Sia $L$ un insieme aperto di rette. Allora esiste un insieme boreliano di punti $A$ per cui:\\
		\begin{itemize}
			\item ogni retta di $L$ interseca $A$ in un insieme residuo;\\
			\item in ogni direzione ci sono trascurabili linee che intersecano $A$ e non appartengono a $L$.\\
		\end{itemize}
	\end{lemma}
	
	\pause
	Si costruisce $A$ tramite intersezioni e unioni di parallelogrammi.\\ \pause
	La condizione topologica è piuttosto debole.\\ \pause
	Presa una linea $\ell \not \in L$ che inteseca $A$, mostro che la proiezione di $A$ in direzione di $\ell$ è arbitrariamente piccola.
	
	
\end{frame}	


\begin{frame}[fragile]
	\frame{Costruzioni per parallelogrammi}
	
	\begin{center}
	\includegraphics[width=0.5\columnwidth]{passi123.png}
	\end{center}

	Induttivamente, all'$n$-esimo passo, una retta in direzione in $D_1$ che interseca $P$, interseca uno dei parallelogrammi dell'$n$-esimo passo.\\
	Inoltre la misura della proiezione dell'unione dei parallelogrammi 
	sulla retta $A_{1} A_{4}$ è al più $2 \cdot\left|A_{1} A_{4}\right|$ in ogni direzione in $D_2,...,D_{n+1}$.\\

\end{frame}



\begin{frame}
	\frametitle{Misurabilità di $L^*$}
	Per $\mu$ generica misura boreliana, dato $A$ misurabile, si trova $L$ boreliano che soddisfa le richieste della tesi. Vediamo perché $L^*$ è misurabile. \pause

\begin{defin}
	Sia $X$ un insieme non vuoto e sia $\mathcal{E}$ una collezione di suoi sottoinsiemi. \pause Diciamo che un insieme $A$ è analitico o di Suslin se è nella forma
	$$
	A=\bigcup_{\left(n_{i}\right) \in \mathbb{N}^{\infty}} \bigcap_{k=1}^{\infty} A_{n_{1}, \ldots, n_{k}} .
	$$
	per opportuni $\left\{A_{n_{1}, \ldots, n_{k}}\right\} \in \E$.\\ \pause
	Questa è detta A-operazione (o operazione di Suslin).\\
	La collezione degli insiemi di questo tipo e l'insieme vuoto è indicata con $S(\mathcal{E})$.
\end{defin}	

\end{frame}

\begin{frame}[fragile]
\frametitle{Misurabilità di $L^*$}	
Valgono i seguenti teoremi generali. \pause

\begin{teo}
	\begin{enumerate}[i]
		\item $S(S(\mathcal{E}))=S(\mathcal{E})$\\ \pause
		\item Se $\E$ chiuso per complementare e $\varnothing \in \E$, la $\sigma$-algebra $\sigma(\mathcal{E})$ generata da $\mathcal{E}$ è contenuta in $S(\mathcal{E}).$
	\end{enumerate}
\end{teo}
\pause
Siccome $L$ insieme boreliano, applicando il teorema con $\E$ la classe dei compatti di $\P^2 \R$, si ha
$$L=\bigcup_{{\bf n}\in{\bf N}^\infty}\bigcap_{k=1}^\infty K_{n_1,\ldots,n_k}$$
con $K_{n_1,\ldots,n_k}\in {\cal E}$.\\ \pause
Si mostra che vale anche
$$L^*=\bigcup_{{\bf n}\in{\bf N}^\infty}\bigcap_{k=1}^\infty K_{n_1,\ldots,n_k}^*.$$

\end{frame}

\begin{frame}
\frametitle{Misurabilità di $L^*$}	
Abbiamo quindi che $L^*$ analitico. Concludiamo applicando il seguente teorema. \\
\pause
\begin{teo}
 	Detta $\mu$ una misura $\sigma$-finita. Se $\E$ famiglia di insiemi misurabili chiusa per unioni finite e intersezioni numerabili, ogni insieme in $S(E)$ è $\mu$-misurabile.
\end{teo}	
	
	
\end{frame}


\begin{frame}[fragile]
\frametitle{Conclusione: generalizzazione del teorema}

\begin{teo}
	Sia $A \subset \mathbb{R}^{2}$ misurabile e $\mu$ una misura boreliana $\sigma$-finita nel piano. Allora esiste un insieme Boreliano di rette $L$ tale che per ogni punto di $A$, l'insieme delle direzioni delle linee di $L$ contenenti il punto sia residuo e $\mu(A)=\mu(L^*)$. \\ \pause
	Similmente, data una misura boreliana $\mu$ $\sigma$-finita sull'insieme delle rette nel piano, per un insieme misurabile di rette $L$, esiste un insieme di punti $A$ tale che ogni retta di $L$ intersechi $A$ in un insieme residuo e $\mu(A^*)=\mu(L)$.
\end{teo}
	\pause

\structure{\textbf{Dimostrazione}}
	\begin{itemize}
		\item Passo 1: possiamo assumere che $\mu$ abbia supporto compatto $K$ disgiunto da $A$.\\
		\item Passo 2: possiamo assumere che $A=B(x, r) \backslash\{x\}$ e $B(x, 2 r)$ sia disgiunta da $K$.\\
		\item Passo 3: applichiamo il lemma per concludere.
	\end{itemize}
\end{frame}


\begin{frame}
\frametitle{Passo 3}
Per brevità, assumiamo che $A=B(0,1) \backslash\{0\}$ e $B(0,2) \cap K=\varnothing$.\\ \pause
Applicando il lemma ad $A$ e $x=0$, otteniamo l'insieme di rette $M$.\\ \pause Sia $M^{* *}=M^{*} \backslash B(0,1)$. Consideriamo lo spazio
$$
\left(\mathbb{R}^{2}, \mu\right) \times(\mathbb{R}, \lambda),
$$
dove $\lambda$ è la misura di Lebesgue sulla retta e consideriamo il sottoinsieme
$$
\left\{(x, t) \in \mathbb{R}^{2} \times \mathbb{R}: t x \in M^{* *}\right\} .
$$ \pause
Per costruzione di $M$, per ogni $x \neq 0$ vale $\lambda(\{t: tx \in M^{**}\})=0$ (tutte le sezioni verticali hanno misura di Lebesgue nulla).\\ \pause
Quasi ogni sezione orizzontale ha misura nulla e possiamo scegliere un numero $u$ tale che $1 / 2<u<1$ e $\mu\left(\left\{x: u x \in M^{* *}\right\}\right)=0$.\\ \pause
Stiamo usando la versione rafforzata del teorema: infatti se $\mu$-quasi ogni sezione verticale ha misura nulla per $\mu$ di Lebesgue, non è detto che questo valga per ogni misura.\\

\end{frame}

\begin{frame}
\frametitle{Passo 3}
Poniamo $t=1 / u$.\\ 
Quindi $1<t<2$ e $\mu\left(t M^{* *}\right)=0$. Affermiamo che $t M$ soddisfa la tesi.\\ \pause
\begin{itemize}
	\item siccome $t M$ contiene residue rette per i punti di $t B(0,1) \backslash\{0\}$ e $t B(0,1) \backslash\{0\} \supset B(0,1) \backslash\{0\}=A$, è vera la prima affermazione.\\
	\item poiché $B(0,2) \cap K=\varnothing$ e $t<2$,
	$$ \mu\left(t M^{*}\right)=\mu\left(t M^{*} \cap K\right)=\mu\left(t M^{*} \backslash B(0, t)\right) $$
	Inoltre
	$$ t M^{*} \backslash B(0, t)=t\left(M^{*} \backslash B(0,1)\right)=t M^{* *} $$
	Siccome $\mu\left(t M^{* *}\right)=0$, anche $\mu (tM^*)= 0= \mu(A)$.
\end{itemize}
\end{frame}


\end{document}%%%%%%%%%%%%%%%%%%%%%%%%
