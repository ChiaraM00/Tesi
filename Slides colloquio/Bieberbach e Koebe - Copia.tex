\documentclass[11pt]{beamer} %handout
\usepackage[utf8]{inputenc} 
\usetheme{Boadilla}
\usepackage{amsmath}
\usepackage{physics}
\usepackage{tikz-cd}
\usepackage{tikz-cd}
\usepackage{adjustbox}
\usepackage{amsmath,amsthm,amssymb,mathtools,amsfonts}
\usepackage[italian]{babel}
\usepackage{pgfornament}

\author{Chiara Molinari}
\title{Costruzioni paradossali in Teoria della Misura: il Teorema di Davies}
\date{29 aprile 2022}

\newcommand{\ra}{\rightarrow}
\newcommand{\Ra}{\Rightarrow}
\newcommand{\C}{\mathbb{C}}
\newcommand{\R}{\mathbb{R}}
\newcommand{\N}{\mathbb{N}}
\newcommand{\Q}{\mathbb{Q}}
\newcommand{\Z}{\mathbb{Z}}
\renewcommand{\S}{\mathbb{S}}
\renewcommand{\P}{\mathbb{P}}
\newcommand{\parti}{\mathcal{P}}
\newcommand{\E}{\mathcal{E}}
\newcommand{\A}{\mathcal{A}}
\newcommand{\fa}{\forall}
\newcommand{\fo}{\forall}
\newcommand{\e}{\varepsilon}
\newcommand{\<}{\langle}
\renewcommand{\>}{\rangle}

% Bold math in titles
\makeatletter%
\DeclareRobustCommand*{\bfseries}{%
	\not@math@alphabet\bfseries\mathbf
	\fontseries\bfdefault\selectfont
	\boldmath
}
\makeatother

\theoremstyle{theorem}
\newtheorem{teo}{Teorema}

\theoremstyle{theorem}
\newtheorem{teorema}{Teorema}

\theoremstyle{theorem}
\newtheorem{proposizione}{Proposizione}

\theoremstyle{theorem}
\newtheorem{corollario}{Corollario}

\newtheorem{defin}[teo]{Definizione}

\theoremstyle{theorem}
\newtheorem{lem}{Lemma}

\usefonttheme{professionalfonts} %questo coso diabilita la gestione di font di beamer




\begin{document}%%%%%%%%%%%%%%%%%%%%%%%%

\begin{frame}
	\maketitle
\end{frame}


\begin{frame}[fragile]
	\structure{\textbf{Teorema (Davies, 1951):}} dato un insieme misurabile $A \subseteq \R^2$, con $\mu(E)< \infty$ (misura di Lebesgue), esiste un insieme di linee $L$ tale che:\\
	(i) per ogni punto di $E$ passi una linea di $L$;\\
	(ii) $\mu(L^*)=\mu(E)$;\\
	dove con $L^*$ si indica il sottoinsieme di $\R^2$ coperto dalle rette di $L$.
	
	\medskip
	
	Nel 2001, M. Csornyei dimostra un rafforzamento del teorema e una sua generalizzazione al caso di $\mu$ generica misura boreliana $\sigma$-finita. 
	
	\medskip
	 
	In questo colloquio:\\
	\begin{itemize}
		\item accenniamo alla dimostrazione originale;\\
		\item mostriamo alcune idee utilizzate da Csornyei per dimostrare il teorema rafforzato (la costruizione geometrica "venetian blind" e la formulazione duale del problema);\\
		\item tratteremo il problema della misurabilità di $L^*$ come sopra;\\
		\item dimostreremo il caso per $\mu$ generica boreliana $\sigma$-finita.
	\end{itemize}

\end{frame}
	
\begin{frame}
	
	
	La dimostrazione originale utilizzava il seguente lemma.\\
	
	Sia $R$ un parallelogramma $R$ e $K$ un sottoinsieme chiuso (misurabile). Allora dato $\varepsilon>0$, esiste una collezioni finita (numerabile) di parallelogrammi $\{P_i\}_{i \in I}$ contenuti in $R$, con lati paralleli a quelli di $R$ tali che:\\
	\begin{enumerate}[i]
		\item $K \subseteq \bigcup_i P_i$;	
		\item $|\bigcup_i P_i \setminus K|< \varepsilon$.
	\end{enumerate}
	

\end{frame}



\begin{frame}[fragile]
	Per enunciare il rafforzamento di Csornyei, diamo prima la seguente definizione.\\
	
	\begin{defin}
	Un insieme è della prima categoria se unione numerabile di insiemi mai densi (cioè con chiusura a parte interna vuota).\\
	Un insieme è residuo se il suo complementare è della prima categoria. 
	\end{defin}

\end{frame}

\begin{frame}[fragile]

\begin{lemma}
	Sia $A$ aperto del piano e sia $x$ un punto che non appartiene ad $A$. Allora esiste un insieme boreliano di rette $L$ tale che:\\
	\begin{itemize}
		\item L contiene un insieme residuo di rette per ogni punto di $A$;\\
		\item $L^{*} \backslash A$ interseca ogni retta per $x$ in un insieme di misura di Lebesgue nulla.\\
	\end{itemize}
\end{lemma}


	Vediamo che è un rafforzamento.\\
	\begin{itemize}
		\item Si mostra che condizione $A$ non è restrittiva.\\
		\item Ricordiamo un insieme è di misura nulla se e solo se interseca $quasi$ ogni retta in un insieme di misura lineare nulla. Togliendo il $quasi$, la tesi è più forte.\\
		\item Aggiungiamo una condizione topologica su $L$.
	\end{itemize}

\end{frame}

\begin{frame}[fragile]

Formulazione duale del problema.
Esiste una corrispondenza naturale tra le rette in $\P^2 \R$ e i punti in $\P^2\R$: alla retta di equazione $ax_0+bx_1+cx_2=0$ associo $[a,b,c]$.\\
Se una retta è affine, considero la sua equazione omogeneizzata.\\
Si consideri il piano proiettivo reale $\P^2 \R$, realizzato come quoziente di $\S^2$. La naturale misura di area su $\S^2$ induce tramite $\pi$ una misura su $\P^2\R$.\\
Così possiamo definire una misura sulle rette.
\end{frame}

\begin{frame}
	Enunciamo così il la versione duale del precedente lemma.
	
	\begin{lemma}
	Sia $L$ un insieme aperto di rette (cioè corrispondente nel duale a un aperto di $\P^2 \R$)e sia $X$ un retta non appartenente a $L$. Allora esiste un insieme boreliano di punti $A$ per cui:\\
	\begin{itemize}
		\item ogni retta di $L$ interseca $A$ in un insieme residuo;\\
		\item per ogni punto di $X$ passano trascurabili rette di $A^{*} \backslash L$.\\
	\end{itemize}
	Con $A^*$ indichiamo l'insieme delle rette passanti per i punti di $A$.
	\end{lemma}
	
\end{frame}	



\begin{frame}[fragile]
	Risolta difficile dare un'idea intuitiva della dimostrazione di questo lemma senza passare per teoremi tecnici.\\
	
	A titolo di esempio, vediamo la costruzione geometrica detta "venetian blind", che dato un parallelogramma $P$, gli associa una collezione di parallelogrammi...
	
\end{frame}


\begin{frame}[fragile]

Per $\mu$ generica misura boreliana, dato $A$ misurabile, si trova $L$ boreliano che soddisfa le richieste della tesi. Perché $L^*$ è misurabile?

\begin{defin}
	Sia $X$ un insieme non vuoto e sia $\mathcal{E}$ una collezione di suoi sottoinsiemi. Diciamo che un insieme $A$ è analitico o di Suslin se è nella forma
	$$
	A=\bigcup_{\left(n_{i}\right) \in \mathbb{N}^{\infty}} \bigcap_{k=1}^{\infty} A_{n_{1}, \ldots, n_{k}} .
	$$
	per opportuni $\left\{A_{n_{1}, \ldots, n_{k}}\right\} \in \E$.\\
	Questa è detta A-operazione (o operazione di Suslin).\\
	La collezione degli insiemi di questo tipo e l'insieme vuoto è indicata con $S(\mathcal{E})$.
\end{defin}	

\end{frame}

\begin{frame}[fragile]
Valgono i seguenti teoremi generali.

\begin{teo}
	\begin{enumerate}[i]
		\item $S(S(\mathcal{E}))=S(\mathcal{E})$\\
		\item Sotto opportune ipotesi, la $\sigma$-algebra $\sigma(\mathcal{E})$ generata da $\mathcal{E}$ è contenuta in $S(\mathcal{E}).$
	\end{enumerate}
\end{teo}

Applicando il teorema con $\E$ la classe dei compatti di $\P^2 \R$, si ha
$$L=\bigcup_{{\bf n}\in{\bf N}^\infty}\bigcap_{k=1}^\infty K_{n_1,\ldots,n_k}$$
con $K_{n_1,\ldots,n_k}\in {\cal E}$.\\
Si mostra che vale anche
$$L^*=\bigcup_{{\bf n}\in{\bf N}^\infty}\bigcap_{k=1}^\infty K_{n_1,\ldots,n_k}^*.$$ 

\end{frame}

\begin{frame}
Abbiamo quindi che $L^*$ analitico. Concludiamo applicando il seguente teorema. \\

\begin{teo}
 	Detta $\mu$ una misura $\sigma$-finita. Per opportune ipotesi su $\E$, ogni $\mathcal{E}$-insieme di Souslin è $\mu$-misurabile.
\end{teo}	
	
	
\end{frame}


\begin{frame}[fragile]

Si mostra che possiamo assumere che $\mu$ abbia supporto compatto $K$ disgiunto da $A$.\\
Quindi possiamo assumere che $A=B(x, r) \backslash\{x\}$ e $B(x, 2 r)$ sia disgiunta da $K$.\\
Per brevità, assumiamo che $A=B(0,1) \backslash\{0\}$ e $B(0,2) \cap K=\varnothing$.\\
Applichiamo il lemma ad $A$ e $x=0$, e sia $M$ l'insieme di rette così ottenuto.\\
Sia $M^{* *}=M^{*} \backslash B(0,1)$. Consideriamo lo spazio
$$
\left(\mathbb{R}^{2}, \mu\right) \times(\mathbb{R}, \lambda),
$$
dove $\lambda$ è la misura di Lebesgue sulla retta e consideriamo il sottoinsieme
$$
\left\{(x, t) \in \mathbb{R}^{2} \times \mathbb{R}: t x \in M^{* *}\right\} .
$$
Poiché $M$ soddisfa la seconda condizione del lemma 2, $M^{**}$ interseca ogni retta per l'origine in un insieme di misura di Lebesgue nulla, cioè per ogni $x \neq 0$ vale $\lambda(\{t: tx \in M^{**}\})=0$ (tutte le sezioni verticali di questo sottoinsieme hanno misura di Lebesgue nulla). Quindi quasi ogni sezione orizzontale ha misura nulla e possiamo scegliere un numero $u$ tale che $1 / 2<u<1$ e $\mu\left(\left\{x: u x \in M^{* *}\right\}\right)=0$, e porre $t=1 / u$. Quindi $1<t<2$ e $\mu\left(t M^{* *}\right)=0$. Affermiamo che $t M$ soddisfa la tesi.\\
\end{frame}


\begin{frame}
Siccome $M$ contiene residue rette per ogni punto di $B(0,1) \backslash\{0\}$, $t M$ contiene residue rette per i punti di $t B(0,1) \backslash\{0\}$, ma $t B(0,1) \backslash\{0\} \supset B(0,1) \backslash\{0\}=A$ quindi è vera la prima affermazione.\\
D'altra parte, siccome $B(0,2) \cap K=\varnothing$ e $t<2$, si ha
$$
\mu\left(t M^{*}\right)=\mu\left(t M^{*} \cap K\right)=\mu\left(t M^{*} \backslash B(0, t)\right)
$$
Inoltre
$$
t M^{*} \backslash B(0, t)=t\left(M^{*} \backslash B(0,1)\right)=t M^{* *}
$$
Siccome $\mu\left(t M^{* *}\right)=0$, anche $\mu (tM^*)= 0= \mu(A)$.

\end{frame}


\end{document}%%%%%%%%%%%%%%%%%%%%%%%%
